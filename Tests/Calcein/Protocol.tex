\documentclass[10pt,a4paper]{Protocol}

% Change the page layout if you need to
\geometry{left=1cm,right=9cm,marginparwidth=6.8cm,marginparsep=1.2cm,top=1cm,bottom=1cm}

% Change the font if you want to.

% If using pdflatex:
\usepackage[utf8]{inputenc}
\usepackage[T1]{fontenc}
\usepackage[default]{lato}
\usepackage{biblatex}
\usepackage{ragged2e}
\usepackage{xcolor}


% If using xelatex or lualatex:
\setmainfont{Lato}

% Change the colours if you want to
\definecolor{Blue}{RGB}{0,0,255}
\definecolor{Green}{RGB}{0,255,0}
\definecolor{Red}{RGB}{255,0,0}
\definecolor{Black}{RGB}{0,0,0}
\colorlet{heading}{Red}
\colorlet{accent}{Blue}
\colorlet{emphasis}{Black}
\colorlet{body}{Black}

% Change the bullets for itemize and rating marker
% for \risk if you want to
\renewcommand{\itemmarker}{{\small\textbullet}}
\renewcommand{\ratingmarker}{\faSpinner}

%% sample.bib contains your publications
\addbibresource{Bibliography.bib}

\begin{document}
\name{Calcein Staining}
\tagline{Protocol used for calcein bulk staining of microcracks}
\made{September 29, 2021}
\logo{6.5cm}{"Unibe_Logo"}


\docinfo{%
  % can add more \addedto{people}
  \madeby{Silvia Owusu}{silvia.owusu@zmk.unibe.ch}{September 29 2021}
  \madeby{Mathieu Simon}{mathieu.simon@artorg.unibe.ch}{September 29 2021}
}


\purpose{This protocol is aimed to stain in vivo microcracks by performing bulk staining of 3 mm thick bone samples using calcein. Calcein binds to free calcium present at the surface of microcracks and bulk staining allow to stain microcracks present in vivo only, not the ones generated by sample preparation.} % add a short discription of the purpose for this protocol


%% Make the header extend all the way to the right, if you want.
\begin{fullwidth}
\makeheader
\end{fullwidth}

%% Provide the file name containing the sidebar contents as an optional parameter to \need.
%% You can always just use \marginpar{...} if you do
%% not need to align the top of the contents to any
%% \need title in the "main" bar.
\need[Info]{Protocol}

\justifying
\step{1}{Cutting}{? + ?}
\begin{itemize}
	\item Cut bone sample using a butcher saw into slices of 3 mm thickness.
\end{itemize}
\divider

\step{2}{Fixation}{? + ?}
\begin{itemize}
	\item Do something
\end{itemize}
\divider

\step{3}{Bulk staining}{? + ?}
\begin{itemize}
	\item Dissolve calcein at a concentration of 0.5 mM. Dissolution is performed either in distilled water or in 100\% ethanol.
	\item Perform bulk staining under low vacuum (-300 mbar) at room temperature. 2 samples are stained for 7 hour and 2 are for 24 hours.
\end{itemize}
\divider

\step{4}{Embedding}{? + ?}
\begin{itemize}
	\item Do something
\end{itemize}
\divider

\step{5}{Sample preparation}{? + ?}
\begin{itemize}
	\item Do something
\end{itemize}
\divider

\step{6}{Histology analysis}{? + ?}
\begin{itemize}
	\item Do something
\end{itemize}



%% If the NEXT page doesn't start with a \need but you'd
%% still like to add a sidebar, then use this command on THIS
%% page to add it. The optional argument lets you pull up the
%% sidebar a bit so that it looks aligned with the top of the
%% main column.
% \addnextpagesidebar[-1ex]{page3sidebar}


\end{document}
